\documentclass{article}

\usepackage{amsmath}
\usepackage{amsfonts}
\usepackage{amssymb}
\usepackage{multicol}
\usepackage{wrapfig}
\usepackage{mathenv}
\usepackage{multirow}
\usepackage{pdfpages}
\usepackage{vmargin}
\setmarginsrb{2.5cm}{2.5cm}{2.5cm}{2.9cm}{0cm}{0cm}{0cm}{0cm}

\usepackage[utf8]{inputenc}

\usepackage[french]{babel}
\selectlanguage{french}

\usepackage{color}
\usepackage{hyperref}
\hypersetup{pdfborder={0 0 0}, colorlinks=true, urlcolor=blue, linkcolor = darkred}
\usepackage{graphicx}
\graphicspath{{pdf/}} 
\usepackage{listings}
\definecolor{colKeys}{rgb}{0.75,0,0}
\definecolor{colIdentifier}{rgb}{0,0,0}
\definecolor{colComments}{rgb}{0.75,0.75,0}
\definecolor{colString}{rgb}{0,0,0.7}

\usepackage{verbatim}
\usepackage{moreverb}

% Commandes personnelles %

\definecolor{darkred}{rgb}{0.85,0,0}
\definecolor{darkblue}{rgb}{0,0,0.7}
\definecolor{darkgreen}{rgb}{0,0.6,0}
\definecolor{darko}{rgb}{0.93,0.43,0}
\definecolor{maintitle}{rgb}{0.66,0,0.22}
\definecolor{title}{rgb}{0,0.5,0.5}
\definecolor{quote}{rgb}{0.7,0.7,0.7}
\definecolor{forestgreen}{rgb}{0.14,0.54,0.13}
\definecolor{cyan4}{rgb}{0,0.54,0.54}
\definecolor{firebrick4}{rgb}{0.54,0.1,0.1}
\newcommand{\maintitlecolor}[1]{\textcolor{maintitle}{#1}}
\newcommand{\titre}[1]{\textcolor{title}{#1}}
\newcommand{\tsect}[1]{\titre{\section{#1}}}
\newcommand{\tssect}[1]{\titre{\subsection{#1}}}
\newcommand{\tsssect}[1]{\titre{\subsubsection{#1}}}
\newcommand{\vect}[1]{\overrightarrow{#1}}
\newcommand{\dred}[1]{\textcolor{darkred}{\textbf{#1}}}
\newcommand{\dgre}[1]{\textcolor{darkgreen}{\textbf{#1}}}
\newcommand{\dblu}[1]{\textcolor{darkblue}{\textbf{#1}}}
\newcommand{\dora}[1]{\textcolor{darko}{\textbf{#1}}}
\newcommand{\gre}[1]{\textcolor{darkgreen}{#1}}
\newcommand{\blu}[1]{\textcolor{darkblue}{#1}}
\newcommand{\ora}[1]{\textcolor{darko}{#1}}
\newcommand{\rouge}[1]{\textcolor{darkred}{#1}}
\newcommand{\quotecolor}[1]{\textcolor{quote}{#1}}
\newcommand{\forest}[1]{\textcolor{forestgreen}{#1}}
\newcommand{\cyan}[1]{\textcolor{cyan4}{#1}}
\newcommand{\firebrick}[1]{\textcolor{firebrick4}{#1}}
\newcommand{\ceil}[1]{\left\lceil #1 \right\rceil}
\newcommand{\cdil}[1]{\left\lfloor #1 \right\rfloor}
\newcommand{\term}[1]{\textit{\textcolor{maintitle}{#1}}}
\newcommand{\image}[1]{\includegraphics{#1}}
\newcommand{\imageR}[2]{\includegraphics[width=#2px]{#1}}
\newcommand{\imageRT}[2]{\includegraphics[height=#2px]{#1}}
\newcommand{\img}[1]{\begin{center}\includegraphics[width=400px]{#1}\end{center}}
\newcommand{\imag}[1]{\begin{center}\includegraphics{#1}\end{center}}
\newcommand{\imgR}[2]{\begin{center}\includegraphics[width=#2px]{#1}\end{center}}
\newcommand{\imgRT}[2]{\begin{center}\includegraphics[height=#2px]{#1}\end{center}}
\newcommand{\point}[2]{\item \ora{\underline{#1}} : \textit{#2}}
\newcommand{\bfp}[2]{\item \textbf{#1} : \textit{#2}}
\newcommand{\sumparam}[3]{\sideset{}{_{#1}^{#2}}\sum{#3}}
\newcommand{\sumin}[3]{\sideset{}{_{i=#1}^{#2}}\sum{#3}}
\newcommand{\sumkn}[3]{\sideset{}{_{k=#1}^{#2}}\sum{#3}}
\newcommand{\intin}[3]{\sideset{}{_{#1}^{#2}}\int{#3}}
\newcommand{\stitre}[1]{\noindent\textbf{\underline{#1}} \\}
\newcommand{\R}{\mathbb{R}}
\newcommand{\Z}{\mathbb{Z}}
\newcommand{\N}{\mathbb{N}}
\newcommand{\ualpha}{\vect{u_\alpha}}
\newcommand{\valpha}{\vect{v_\alpha}}
\newcommand{\palpha}{\vect{\Psi_\alpha}}
\newcommand{\npcomp}{\term{$\mathcal{NP}$-complet}}
\newcommand{\npcompl}{\term{$\mathcal{NP}$-complet} }
\newcommand{\cqfd}{\begin{flushright}$\square$\end{flushright}}
\newcommand{\contrad}{\begin{flushright}$\boxtimes$\end{flushright}}
\DeclareMathAlphabet{\mathpzc}{OT1}{pzc}{m}{it}
\newtheorem{de}{D\'efinition}[section]
\newtheorem{note}{Note}[section]
\newtheorem{propriete}{Propri\'et\'e}[section]
\newtheorem{exemple}{Exemple}[section]
\newtheorem{corollaire}{Corollaire}[section]
\newtheorem{interlude}{Interlude}[section]
\newtheorem{rappel}{Rappel}[section]
\newtheorem{rem}{Remarque}[section]
\newtheorem{rems}{Remarques}[section]
\newtheorem{thm}{Th\'eor\`eme}[section]
\newtheorem{lemme}{Lemme}[section]
\newtheorem{illustration}{Illustration}[section]
\newtheorem{pbm}{Problème}[section]
\newtheorem{proof}{Preuve}[section]
\renewcommand{\theproof}{\empty{}} 
\newenvironment{pblm}{\hbox{\raisebox{0.4em}{\vrule depth 1pt height 0.4pt width 5cm}}\begin{pbm}}
{\end{pbm}\hbox{\raisebox{0.4em}{\vrule depth 1pt height 0.4pt width 5cm}}}


%%%%%%%%%%%%%%%%%%%%%%%%%%%%%%%%%%%%%%%%%%%%%%%%%%%%%%%%%%%%%%%%%%%
%%%%%%%%%%%%%%%%%%%%%%%% DEBUT DU DOCUMENT %%%%%%%%%%%%%%%%%%%%%%%%
%%%%%%%%%%%%%%%%%%%%%%%%%%%%%%%%%%%%%%%%%%%%%%%%%%%%%%%%%%%%%%%%%%%

\begin{sffamily}
\title{$ $\\
\hbox{\raisebox{0.4em}{\vrule depth 2pt height 0.4pt width \textwidth}} $ $ \\ $ $ \\ $ $\\$ $\\$ $\\$ $\\
\begin{Huge}\maintitlecolor{Stage en entreprise \\$ $\\ \includegraphics[scale=0.8]{logo.pdf}}\end{Huge} \\ 
$ $ \\ 
\begin{LARGE}\textit{Cahier des charges}\end{LARGE}}
\author{\textit{Xavier Dubuc} \\(\url{xavier.dubuc@umons.ac.be}) \\$ $ \\$ $\\$ $\\$ $\\$ $\\$ $\\$ $\\$ $\\$ $\\$ $\\
\hbox{\raisebox{0.4em}{\vrule depth 1pt height 0.4pt width 5cm}} }
%\date{}
\end{sffamily}

\begin{document}\begin{sffamily}

\maketitle

\newpage

\section*{Création d'une application de génération et gestion d'affiches électorales pour la FGTB}

\begin{enumerate}
\item \textbf{Planification du projet}
	\begin{enumerate}
		\item[1.1.] Etablissement du cahier des charges
		\item[1.2.] Délimitation des tâches
		\item[1.3.] Passage en revue des techniques et langages à utiliser
		\item[1.4.] Etablissement du schéma de base de données
	\end{enumerate}
\item \textbf{Apprentissage }
	\begin{enumerate}
		\item[2.1.] Court rappel du langage \textbf{PHP} et de \includegraphics[scale=0.05]{git.pdf} (Branching Model)
		\item[2.2.] Apprentissage et expérimentation de \textbf{Symfony 2}
		\item[2.3.] Apprentissage et expérimentation de \textbf{javascript} et \textbf{jQuery}
	\end{enumerate}
\item \textbf{Implémentation Feature \textit{``user''}} \includegraphics[scale=0.5]{symfony.pdf} \includegraphics[scale=0.15]{php.pdf}
	\begin{enumerate}
		\item[3.1.] Fonctionnalité ``Créer un compte''
		\item[3.2.] Fonctionnalité ``Récupération de l'identifiant''
		\item[3.3.] Fonctionnalité ``Login/Logout''	
		\item[3.4.] Fonctionnalité ``Modification du compte''	
	\end{enumerate}
\item \textbf{Implémentation Feature \textit{``pictures''}} \includegraphics[scale=0.5]{symfony.pdf}\includegraphics[scale=0.2]{jquery.pdf}\includegraphics[scale=0.15]{php.pdf}
	\begin{enumerate}
		\item[4.1.] Fonctionnalité ``Upload d'images''
		\item[4.2.] Fonctionnalité ``Affichage des images uploadées''
		\item[4.3.] Fonctionnalité ``Génération de miniatures''
		\item[4.4.] Fonctionnalité ``Suppression d'une image''
	\end{enumerate}
\item \textbf{Implémentation Feature \textbf{``compositing''}} \includegraphics[scale=0.5]{symfony.pdf}\includegraphics[scale=0.2]{jquery.pdf}\includegraphics[scale=0.15]{php.pdf}
	\begin{enumerate}
		\item[5.1.] Fonctionnalité ``Sélection du slogan''
		\item[5.2.] Fonctionnalité ``Sélection de la photo''
		\item[5.3.] Fonctionnalité ``Sélection du format de l'affiche''
		\item[5.4.] Fonctionnalité ``Sélection du logo''
		\item[] Note : Chaque fonctionnalité met à jour un aperçu de l'affiche en temps réel.
	\end{enumerate}
\item \textbf{Implémentation Feature \textit{``pdf''}} \includegraphics[scale=0.5]{symfony.pdf}\includegraphics[scale=0.2]{jquery.pdf}
	\begin{enumerate}
		\item[6.1] Fonctionnalité ``Enregistrement de l'affiche au format pdf''
		\item[6.2] Fonctionnalité ``Affichage d'une liste des 5 derniers pdf's créés (miniature)''
		\item[6.3] Fonctionnalité ``Suppression d'un pdf''
		\item[6.4] Fonctionnalité ``Téléchargement d'un pdf''
		\item[6.5] Fonctionnalité ``Affichage d'un pdf''
	\end{enumerate}
\item \textbf{Mise en place du design} \includegraphics[scale=0.2]{jquery.pdf} \includegraphics[scale=0.075]{html.pdf} \includegraphics[scale=0.01]{css.pdf}
\end{enumerate}

\end{sffamily}\end{document}