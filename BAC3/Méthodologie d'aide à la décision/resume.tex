\documentclass{article}

\usepackage{amsmath}
\usepackage{amsfonts}
\usepackage{amssymb}
\usepackage{multicol}
\usepackage{mathrsfs}
\usepackage{mathenv}

\def\nbOne{{\mathchoice {\rm 1\mskip-4mu l} {\rm 1\mskip-4mu l}
{\rm 1\mskip-4.5mu l} {\rm 1\mskip-5mu l}}}

\usepackage{vmargin}
\setmarginsrb{2.5cm}{2.5cm}{2.5cm}{2.5cm}{0cm}{0cm}{0cm}{0cm}

\usepackage[utf8]{inputenc}

\usepackage[french]{babel}
\selectlanguage{french}

\usepackage{color}
\usepackage{graphicx}
\graphicspath{{img/}} 

\usepackage{verbatim}
\usepackage{moreverb}

% Commandes personnelles %

\definecolor{darkred}{rgb}{0.85,0,0}
\definecolor{darkblue}{rgb}{0,0,0.7}
\definecolor{darkgreen}{rgb}{0,0.6,0}
\definecolor{darko}{rgb}{0.93,0.43,0}
\newcommand{\dred}[1]{\textcolor{darkred}{\textbf{#1}}}
\newcommand{\dgre}[1]{\textcolor{darkgreen}{\textbf{#1}}}
\newcommand{\dblu}[1]{\textcolor{darkblue}{\textbf{#1}}}
\newcommand{\dora}[1]{\textcolor{darko}{\textbf{#1}}}
\newcommand{\gre}[1]{\textcolor{darkgreen}{#1}}
\newcommand{\blu}[1]{\textcolor{darkblue}{#1}}
\newcommand{\ora}[1]{\textcolor{darko}{#1}}
\newcommand{\red}[1]{\textcolor{darkred}{#1}}

\newcommand{\image}[1]{\includegraphics{#1}}
\newcommand{\imageR}[2]{\includegraphics[width=#2px]{#1}}
\newcommand{\imageRT}[2]{\includegraphics[height=#2px]{#1}}
\newcommand{\img}[1]{\begin{center}\includegraphics[width=400px]{#1}\end{center}}
\newcommand{\imag}[1]{\begin{center}\includegraphics{#1}\end{center}}
\newcommand{\imgR}[2]{\begin{center}\includegraphics[width=#2px]{#1}\end{center}}
\newcommand{\imgRT}[2]{\begin{center}\includegraphics[height=#2px]{#1}\end{center}}
\newcommand{\point}[2]{\item \ora{\underline{#1}} : \textit{#2}}
\newcommand{\bfp}[2]{\item \textbf{#1} : \textit{#2}}
\newcommand{\sumin}[3]{\sideset{}{_{i=#1}^{#2}}\sum{#3}}
\newcommand{\stitre}[1]{\noindent\textbf{\underline{#1}}}
\newcommand{\stitreD}[2]{\noindent\textbf{\underline{#1}} \textit{(#2)}\\}

\newcommand{\neu}{n\oe ud}
\newcommand{\neuSP}{n\oe ud }
\newcommand{\neus}{n\oe uds}
\newcommand{\neuSPs}{n\oe uds }

\DeclareMathAlphabet{\mathpzc}{OT1}{pzc}{m}{it}

\title{\textbf{\textcolor{darkblue}{MAD $\sim$ Résumé 2010.}}}
\author{\textit{Dubuc Xavier} \\ \red{xionhearts@hotmail.com}}

\begin{document}

\maketitle

\hbox{\raisebox{0.4em}{\vrule depth 0.4pt height 0.4pt width 10cm}}

\tableofcontents

$ $ \\
\hbox{\raisebox{0.4em}{\vrule depth 0.4pt height 0.4pt width 10cm}}

\newpage

\section{Méthodes basées sur Condorcet}

\textbf{\underline{Exemple} :}
\begin{center}
	\begin{tabular}{|*{4}{c|}}
	\hline
	\textbf{10v }& \textbf{8v} & \textbf{4v} & \textbf{2v} \\
	\hline
	C & B & C & D \\
	B & A & B & A \\
	D & D & A & E \\
	E & E & E & B \\
	A & C & D & C \\
	\hline
	\end{tabular}
\end{center}
\subsection{Relation de Condorcet}

Toutes les méthodes suivantes sont basées sur cette relation.
Dresser le tableau de préférences et si $A\ P\ B$ et on place une flèche dans le graphe de $A$ vers $B$. \\

\stitre{Calcul des préférences} : \\

\noindent $AB : BPA (2-22)$\\
$AC : CPA (10-14)$ \\
$AD : AID (12-12)$ \\
$AE : APE (14-10)$ \\
$BC : CPB (10-14)$ \\
$BD : BPD (22-2)$ \\
$BE : BPE (22-2)$ \\
$CD : CPD (14-10)$ \\
$CE : CPE (14-10)$ \\
$DE : DPE (20-4)$ \\

\textbf{\underline{Tableau des préférences} : }

\begin{center}
	\begin{tabular}{|*{6}{c|}}
	\hline
	& A & B & C & D & E \\
	\hline
	A & - & & & & \red{P} \\
	B & P & - & & P & \red{P} \\
	C & \gre{P} & \gre{P} & - & \gre{P} & \gre{P} \\
	D & & & & - & \red{P} \\
	E & & & & & - \\
	\hline
	\end{tabular}
\end{center}

\noindent On voit que \gre{C} est un vainqueur de \textbf{Condorcet} et que \red{E} est un vainqueur de \textbf{Condorcet}.

\noindent On construit le graphe et on voit que sur le graphe, aucune flèche ne va vers $C$, ce qui signifie que c'est un vainqueur de \textbf{Condorcet} et que toutes les flèches 
vont vers $E$ ce qui implique que E est un perdant de \textbf{Condorcet}.

\subsection{Méthode de Copeland (ou méthode des scores)}

Le score d'un individu est le nombre de flèches sortantes moins le nombre de flèches rentrantes. Ainsi sur l'exemple : \\

\begin{tabular}{l}
$S(A) = -1$ \\
$S(B) = 2$ \\
$S(C) = 4$ \\
$S(D) = -1$ \\
$S(E) = -4$ \\
\hline
$\qquad \ \ = 0$
\end{tabular}

On voit que le premier est $C$ (logique vu que $C$ est un vainqueur de \textbf{Condorcet}). On réitère en enlevant l'élément de plus haut score. Ici on enlève donc C et les flèches 
concernant celui-ci et on recalcule les scores. On obtient ainsi à la fin le classement : 
\newpage
\begin{center}
	\begin{boxedverbatim}
 C
 |
 B
 |
A,D
 |
 E
	\end{boxedverbatim}
\end{center}

\subsection{Méthode de Schwartz}

Elle consiste à effectuer la fermeture transitive du graphe de \textbf{Condorcet}, puis d'enlever les doubles flèches et les boucles. \textit{(Dans l'exemple, rien à ajouter pour la 
fermeture transitive et rien à enlever par la suite)} \\

On enlève le(s) n\oe uds où n'aboutissent aucune flèches et on les place en premier dans le classement ; ensuite on itère avec le graphe dont on a enlevé ces noeuds et les flèches qui 
les concernent. Dans l'exemple on obtient le classement suivant : 

\begin{center}
	\begin{boxedverbatim}
 C
 |
 B
 |
A,D
 |
 E
	\end{boxedverbatim}
\end{center}

\subsection{Méthode de Fishburn}

Cette méthode consiste à associer un ensemble à chaque élément de la manière suivante : 

\[ E-(X) = \{ Y | Y \rightarrow X \}\]

On fait ensuite le graphe de Fishburn où il existe un arc $X \rightarrow Y$ ssi $E-(X) \subsetneq E-(Y)$. On met en premier le(s) éléments auxquels n'aboutissent aucune flèche 
et on itère comme précédemment.

On a donc pour l'exemple : 

\begin{center}
	\begin{tabular}{l}
	$E-(A) = \{B,C\}$ \\
	$E-(B) = \{C\}$ \\
	$E-(C) = \{\}$ \\
	$E-(D) = \{B,C\}$ \\
	$E-(E) = \{A,B,C,D\}$ 
	\end{tabular}
\end{center}

On obtient le classement : 

\begin{center}
	\begin{boxedverbatim}
 C
 |
 B
 |
A,D
 |
 E
	\end{boxedverbatim}
\end{center}

\subsection{Méthode de Raynaud (méthode des oppositions)}

On value les arcs du graphe de \textbf{Condorcet} par la différence entre les préférences pour l'un est l'autre. Pour l'exemple par exemple, la flèche de $B$ vers $A$ 
comportera l'étiquette $22-2$ c'est-à-dire $20$. La méthode de Raynaud consiste à minimiser les oppositions. On calcule donc les oppositions, c'est-à-dire le maximum des poids 
des flèches entrantes sur le graphe. Dans l'exemple : 

\begin{center}
	\begin{tabular}{l}
	$O(A) = 20$ \\
	$O(B) = 4$ \\
	$O(C) = 0$ \\
	$O(D) = 20$ \\
	$O(E) = 20$ \\
	\end{tabular}
\end{center}

On place en tête celui ou ceux ayant le moins d'opposition (donc le $O$ le plus petit), on l'enlève et on itère. On obtient sur l'exemple le classement : 

\begin{center}
	\begin{boxedverbatim}
 C
 |
 B
 |
A,D
 |
 E
	\end{boxedverbatim}
\end{center}

\subsection{Méthode de Perny (fermeture transitive valuée)}

On value de telle sorte que l'on prend le minimum des poids d'un arc sur le chemin allant de $X$ à $Y$. On supprime ensuite les boucles et les doubles flèches de même valeur et on 
remplace les doubles flèches de valeurs différentes par une seule flèche avec comme poids la différence des 2 poids précédents. (Dans notre exemple le pods la flèche $(B,E)$ 
devient $4$ au lieu de $16$) On classe ensuite en premier celui ou ceux auxquels n'aboutissent aucune flèche, on l'enlève du graphe et on itère. On obtient pour l'exemple : 

\begin{center}
	\begin{boxedverbatim}
 C
 |
 B
 |
A,D
 |
 E
	\end{boxedverbatim}
\end{center}

\subsection{Méthode de Borda}

On associe à chaque ligne un rang, de la manière suivante (pour l'exemple) : 

\begin{center}
	\begin{tabular}{|*{5}{c|}}
	\hline
	\textbf{Rang} & \textbf{10v }& \textbf{8v} & \textbf{4v} & \textbf{2v} \\
	\hline
	1 & C & B & C & D \\
	2 & B & A & B & A \\
	3 & D & D & A & E \\
	4 & E & E & E & B \\
	5 & A & C & D & C \\
	\hline
	\end{tabular}
\end{center}

On calcule ensuite les \textit{« scores »} suivant le rang : 

\begin{center}
	\begin{tabular}{l}
	\textbf{A : } $50+16+12+4 = \boxed{82}$  \\
	\textbf{B : } $20+8+8+8 = \boxed{44}$ \\
	\textbf{C : } $10+40+4+10 = \boxed{64}$ \\
	\textbf{D : } $30+24+20+2 = \boxed{76}$ \\
	\textbf{E : } $40+32+16+6 = \boxed{94}$
	\end{tabular}
\end{center}

\newpage

On obtient le classement : 

\begin{center}
	\begin{boxedverbatim}
 B
 |
 C
 |
 D
 |
 A
 |
 E
	\end{boxedverbatim}
\end{center}

\section{Swing Weights}

Avant de faire les swing weights, en fonction des hypothèses de l'énoncé, on trace les droites d'attractivité \textbf{(attention, faire à l'échelle)}. Associer une valeur à chaque sup 
et inf et en fonction de l'échelle ainsi définie, calculer la valeur pour les objets à classer au milieu. On peut maintenant calculer les swing weights. On sait que : 
\[ \boxed{w_1 + w_2 + w_2 = 1}\]

Pour chaque échelle, on regarde la différence de $n_i(inf)$ à $n_i(sup)$ de telle manière que chaque $n_i$ est identique (donc si un des $n_i(sup)$ est à 100 et tous les autres à 
80, il faut regarder la valeur pour 100 dans cette échelle). Pour cette différence, on calcule ensuite le prix (ou autres) nécessaire pour aller d'une borne à l'autre. (par exemple si 
on a $n_1(sup) = 119$ et $n_1(inf) = 94$, on a une différence de $25$ et si on dit dans l'énoncé que pour $10$ il faut $100$ euros, alors il faudra $250$ euros) On a donc un nombre, 
nommons le $x_i$, pour chaque critère. Ensuite on peut calculer que (par exemple pour 3 critères) : 
\[
\boxed{\begin{array}{l}
w_1 = \frac{x_1}{x_2} w_2 \\
w_3 = \frac{x_3}{x_2} w_2
\end{array}} 
\] 

(si $w_2$ est le swing weight du prix)

\noindent Pour l'exemple des écrans LCD, on doit avoir l'égalité suivante : 
\[ \frac{250}{w_1} = \frac{300}{w_2} = \frac{250}{w_3} \Rightarrow \boxed{w_1 = \frac{5}{16}},\ \boxed{w_2 = \frac{6}{16}},\ \boxed{w_3 = \frac{5}{16}}\]

\noindent On calcule ensuite les $p_i$ avec les formules $p_1+p_2+p_3=1$ et $p_i = \frac{w_i}{n_i(Sup_i)-n_i(Inf_i)} k$, dans l'exemple $n_i(Sup_i)-n_i(Inf_i) = 100$ et en 
faisant $p_1+p_2+p_3=1$ on trouve \textbf{k}. (dans l'exemple $k = 100$ et donc $p_i = w_i$) On peut ainsi calculer les attractivité générale des critères :
\begin{center}
	\begin{tabular}{l}
	$Att_g(A) = \sum p_i n_i = \frac{5}{16} 52 + \frac{6}{16} \frac{500}{6} + \frac{5}{16} 0 = \boxed{\frac{760}{16}}$ \\
	$Att_g(B) = \frac{5}{16}0 + \frac{6}{16} 100 + \frac{5}{16} 20 = \boxed{\frac{700}{16}}$ \\
	$Att_g(C) = \frac{5}{16}100 + \frac{6}{16} 0 + \frac{5}{16} 100 = \boxed{\frac{1000}{16}}$ \\
	\end{tabular}
\end{center}

\newpage

\section{MacBeth}

A partir d'un tableau, il faut pouvoir dire que c'est inconsistant ou pas.

\begin{center}
	\begin{tabular}{|*{6}{c|}}
	\hline
	& A & B & C & D & E \\
	\hline
	A & nulle & faible & forte &forte & extrème \\
	\hline
	B & & nulle & med & med & très forte \\
	\hline
	C & & & nulle & très faible & faible \\
	\hline
	D & & & & nulle & faible \\
	\hline
	E & & & & & nulle \\
	\hline
	\end{tabular}
\end{center}

On associe des valeurs à chaque niveau, on commence par mettre à 0 toute la diagonale puis la "seconde diagonale" on mets 1 au plus faible niveau .

\begin{center}
	\begin{tabular}{|*{6}{c|}}
	\hline
	& A & B & C & D & E \\
	\hline
	A & nulle \gre{0} & faible \gre{2}& forte &forte & extrème \\
	\hline
	B & & nulle \gre{0}& med \gre{3}& med & très forte \\
	\hline
	C & & & nulle \gre{0} & très faible \gre{1}& faible \\
	\hline
	D & & & & nulle \gre{0}& faible \gre{2} \\
	\hline
	E & & & & & nulle \gre{0}\\
	\hline
	\end{tabular}
\end{center}

On considère la 2eme diagonale comme diagonale de référence, ensuite pour tout le reste du tableau la valeur c'est la somme des chemins "élémentaires" pour aller d'un endroit à 
l'autre. Donc, par exemple pour aller de $A$ à $E$, la valeur est de $d(A,B) + d(B,C) + d(C,D) + d(D,E)$ (on remplit toujours diagonale par diagonale !). On remplit la 2eme diagonale 
:

\begin{center}
	\begin{tabular}{|*{6}{c|}}
	\hline
	& A & B & C & D & E \\
	\hline
	A & nulle \gre{0} & faible \gre{2}& forte \red{5}&forte & extrème \\
	\hline
	B & & nulle \gre{0}& med \gre{3}& med \red{4} & très forte \\
	\hline
	C & & & nulle \gre{0} & très faible \gre{1}& faible \red{3}\\
	\hline
	D & & & & nulle \gre{0}& faible \gre{2} \\
	\hline
	E & & & & & nulle \gre{0}\\
	\hline
	\end{tabular}
\end{center}

On voit que faible = med, ce qui est un problème, il faut donc le résoudre. Il faut que $d(B,C) \geq d(C,E)+1$. On a : \\
\begin{center}
	\begin{tabular}{l}
	$d(B,C) \geq d(C,E)+1$ \\
	$d(B,C) \geq d(C,D) + d(D,E) + 1$ \\
	$d(B,C) \geq 1+2+1$ \\
	$\boxed{d(B,C) \geq 4}$
	\end{tabular}
\end{center}

$\Rightarrow$ \textbf{on remplace $med$ par 4 au lieu de 3 et on recalcule la diagonale qui posait problème et la suite}.

\red{On peut avoir fort à 7 et fort à 8 par exemple, c'est pas un problème.}

\begin{center}
	\begin{tabular}{|*{6}{c|}}
	\hline
	& A & B & C & D & E \\
	\hline
	A & nulle \gre{0} & faible \gre{2}& forte $\red{\not{5}6}$&forte & extrème \\
	\hline
	B & & nulle \gre{0}& med $\gre{\not{3} 4}$& med$\red{\not{4}5}$ & très forte \\
	\hline
	C & & & nulle \gre{0} & très faible \gre{1}& faible \red{3}\\
	\hline
	D & & & & nulle \gre{0}& faible \gre{2} \\
	\hline
	E & & & & & nulle \gre{0}\\
	\hline
	\end{tabular}
\end{center}

On calcule la diagonale suivante.

\begin{center}
	\begin{tabular}{|*{6}{c|}}
	\hline
	& A & B & C & D & E \\
	\hline
	A & nulle \gre{0} & faible \gre{2}& forte $\red{\not{5}6}$& forte $\ora{7}$ & extrème \\
	\hline
	B & & nulle \gre{0}& med $\gre{\not{3} 4}$& med$\red{\not{4}5}$ & très forte $\ora{7}$\\
	\hline
	C & & & nulle \gre{0} & très faible \gre{1}& faible \red{3}\\
	\hline
	D & & & & nulle \gre{0}& faible \gre{2} \\
	\hline
	E & & & & & nulle \gre{0}\\
	\hline
	\end{tabular}
\end{center}

\textbf{PROBLEME}, on a très forte = forte. Il faut :
\begin{center}
	\begin{tabular}{l}
	$d(B,E) \geq d(A,D) 1$ \\
	$d(B,C)+d(C,D)+d(D,E) \geq d(A,B)+d(B,C)+d(C,D)+1$ \\
	$d(D,E) \geq 3$
	\end{tabular}
\end{center}

\begin{center}
	\begin{tabular}{|*{6}{c|}}
	\hline
	& A & B & C & D & E \\
	\hline
	A & nulle \gre{0} & faible \gre{2}& forte $\red{\not{5}6}$& forte $\ora{7}$ & extrème \\
	\hline
	B & & nulle \gre{0}& med $\gre{\not{3} 4}$& med$\red{\not{4}5}$ & très forte $\ora{\not{7}8}$\\
	\hline
	C & & & nulle \gre{0} & très faible \gre{1}& faible $\red{\not{3} 4}$\\
	\hline
	D & & & & nulle \gre{0}& faible $\gre{\not{2} 3}$ \\
	\hline
	E & & & & & nulle \gre{0}\\
	\hline
	\end{tabular}
\end{center}

On a exactement le même problème qu'avant $d(B,C) = d(C,E)$ alors qu'on a med et faible. Donc même calcul que tantôt et on modifie $d(B,C)$ et on recalcule.

\begin{center}
	\begin{tabular}{|*{6}{c|}}
	\hline
	& A & B & C & D & E \\
	\hline
	A & nulle \gre{0} & faible \gre{2}& forte $\red{\not{5} \not{6} 7}$& forte $\ora{\not{7} 8}$ & extrème \\
	\hline
	B & & nulle \gre{0}& med $\gre{\not{3} \not{4} 5}$& med$\red{\not{4} \not{5} 6}$ & très forte $\ora{\not{7} \not{8} 9}$\\
	\hline
	C & & & nulle \gre{0} & très faible \gre{1}& faible $\red{\not{3} 4}$\\
	\hline
	D & & & & nulle \gre{0}& faible $\gre{\not{2} 3}$ \\
	\hline
	E & & & & & nulle \gre{0}\\
	\hline
	\end{tabular}
\end{center}

On calcule finalement la toute dernière case :

\begin{center}
	\begin{tabular}{|*{6}{c|}}
	\hline
	& A & B & C & D & E \\
	\hline
	A & nulle \gre{0} & faible \gre{2}& forte $\red{\not{5} \not{6} 7}$& forte $\ora{\not{7} 8}$ & extrème $\blu{11}$\\
	\hline
	B & & nulle \gre{0}& med $\gre{\not{3} \not{4} 5}$& med$\red{\not{4} \not{5} 6}$ & très forte $\ora{\not{7} \not{8} 9}$\\
	\hline
	C & & & nulle \gre{0} & très faible \gre{1}& faible $\red{\not{3} 4}$\\
	\hline
	D & & & & nulle \gre{0}& faible $\gre{\not{2} 3}$ \\
	\hline
	E & & & & & nulle \gre{0}\\
	\hline
	\end{tabular}
\end{center}

On peut ensuite classer les $5$ candidats avec la dernière colonne, de cette façon : 
\begin{center}
	\begin{boxedverbatim}
  ^
A-|-11
 -|-
B-|-9
 -|-
 -|-
 -|-
 -|-
C-|-4
D-|-3
 -|-
 -|-
E-|-0
	\end{boxedverbatim}
\end{center}

Il n'y a donc pas d'inconsistance.

\end{document}