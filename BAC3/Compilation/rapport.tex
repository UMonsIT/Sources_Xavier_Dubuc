\documentclass{article}

\usepackage{amsmath}
\usepackage{amsfonts}
\usepackage{amssymb}
\usepackage{algorithm,algorithmic}

\usepackage{vmargin}
\setmarginsrb{2.5cm}{2.5cm}{2.5cm}{2.5cm}{0cm}{0cm}{0cm}{0cm}

\usepackage[utf8]{inputenc}

\usepackage[french]{babel}
\selectlanguage{french}

\usepackage{color}
\definecolor{darkred}{rgb}{0.85,0,0}
\definecolor{darkblue}{rgb}{0,0,0.7}
\definecolor{darkgreen}{rgb}{0,0.6,0}
\definecolor{darko}{rgb}{0.93,0.43,0}
\newcommand{\dred}[1]{\textcolor{darkred}{\textbf{#1}}}
\newcommand{\dgre}[1]{\textcolor{darkgreen}{\textbf{#1}}}
\newcommand{\dblu}[1]{\textcolor{darkblue}{\textbf{#1}}}
\newcommand{\dora}[1]{\textcolor{darko}{\textbf{#1}}}
\newcommand{\gre}[1]{\textcolor{darkgreen}{#1}}
\newcommand{\blu}[1]{\textcolor{darkblue}{#1}}
\newcommand{\ora}[1]{\textcolor{darko}{#1}}
\newcommand{\red}[1]{\textcolor{darkred}{#1}}

\usepackage{listings}
\lstloadlanguages{Java}

\lstset{
    basicstyle=\small,
    keywordstyle=\bfseries,
    commentstyle=\color{green},
    stringstyle=\ttfamily,
    showstringspaces=false,
    frame=tb,
    language=Java,
    defaultdialect=Java}

\usepackage{graphicx}

\title{Conception d'un interpréteur\\Projet de compilation}
\author{Dubuc Xavier \& Vasseur Cyrielle}

\begin{document}

\maketitle
\newpage
\tableofcontents
\newpage
\section{Introduction}
Il nous a été demandé de concevoir un interpréteur pour un langage de programmation de grammaire donnée (cf. Grammaire de base). Les outils utilisés pour la mise en oeuvre de ce projet sont les suivants :\\
\begin{itemize}
\item le langage de programmation \textbf{JFlex} utile à l'analyse lexicale,
\item le langage de programmation \textbf{JCup} utile à l'analyse syntaxique et sémantique.\\
\end{itemize}

Des extensions à cette grammmaire sont demandées telles que la gestion des opérations de multiplication et de division, la lecture en console, la gestion des commentaires, \dots



\section{Grammaire de base}
\noindent Nous allons tout d'abord rappeler la grammaire de base qui, comme nous le verrons dans la section suivante, a été légèrement modifiée.
\begin{center}
\fbox{
\begin{minipage}{0.85\textwidth}
$<$\textit{programme}$>$ $\rightarrow$ $\textbf{debut}$ $<$\textit{liste\_instructions}$>$ \textbf{fin}\\
$<$\textit{liste\_instructions}$>$ $\rightarrow$ $<$\textit{instruction}$>$ $<$\textit{fin\_liste\_instruction}$>$\\
$<$\textit{fin\_liste\_instruction}$>$ $\rightarrow$ $<$\textit{instruction}$>$ $<$\textit{fin\_liste\_instruction}$>$\\
$<$\textit{fin\_liste\_instruction}$>$ $\rightarrow$ $\epsilon$\\
$<$\textit{instruction}$>$ $\rightarrow$ \textbf{ID :=} $<$\textit{expression}$>$ ;\\
$<$\textit{instruction}$>$ $\rightarrow$ \textbf{lire (}$<$\textit{liste\_id}$>$\textbf{) ;}\\
$<$\textit{instruction}$>$ $\rightarrow$ \textbf{ecrire (}$<$\textit{liste\_expressions}$>$\textbf{) ;}\\
$<$\textit{liste\_id}$>$ $\rightarrow$\textbf{ ID} $<$\textit{fin\_liste\_id}$>$\\
$<$\textit{fin\_liste\_id}$>$ $\rightarrow$ \textbf{, ID} $<$\textit{fin\_liste\_id}$>$\\
$<$\textit{fin\_liste\_id}$>$ $\rightarrow$ $\epsilon$\\
$<$\textit{liste\_expressions}$>$ $\rightarrow$ $<$\textit{expression}$>$ $<$\textit{fin\_liste\_expression}$>$\\
$<$\textit{fin\_liste\_expression}$>$ $\rightarrow$ $<$\textit{expression}$>$ $<$\textit{fin\_liste\_expression}$>$\\
$<$\textit{fin\_liste\_expression}$>$ $\rightarrow$ \textbf{,} $<$\textit{expression}$>$ $<$\textit{fin\_liste\_expression}$>$\\
$<$\textit{fin\_liste\_expression}$>$ $\rightarrow$ $\epsilon$\\
$<$\textit{expression}$> \rightarrow$ $<$\textit{expression\_simple}$>$ $<$\textit{fin\_expression\_simple}$>$\\
$<$\textit{fin\_expression\_simple}$>$ $\rightarrow$ $<$\textit{operateur}$>$ $<$\textit{expression\_simple}$>$ $<$\textit{fin\_expression\_simple}$>$\\
$<$\textit{fin\_expression\_simple}$>$ $\rightarrow$ $\epsilon$\\
$<$\textit{expression\_simple}$>$ $\rightarrow$ \textbf{(}$<$\textit{expression}$>$\textbf{)}\\
$<$\textit{expression\_simple}$>$ $\rightarrow$ \textbf{ID} \\
$<$\textit{expression\_simple}$>$ $\rightarrow$ \textbf{INT}\\
$<$\textit{operateur}$>$ $\rightarrow$ \textbf{+}\\
$<$\textit{operateur}$>$ $\rightarrow$ \textbf{-}
\end{minipage}}
\end{center}

\newpage
\section{Grammaire étendue}

\begin{center}
\fbox{
\begin{minipage}{0.8\textwidth}
\noindent $<$\textit{programme}$>$ $\rightarrow$ \textbf{\red{debut}} $<$\textit{liste\_instructions}$>$ \textbf{\red{fin}} \\
$<$\textit{liste\_instructions}$>$ $\rightarrow$ $<$\textit{instruction}$>$ $<$\textit{fin\_liste\_instruction}$>$\\
$<$\textit{fin\_liste\_instruction}$>$ $\rightarrow$ $<$\textit{instruction}$>$ $<$\textit{fin\_liste\_instruction}$>$\\
$<$\textit{fin\_liste\_instruction}$>$ $\rightarrow$ $\epsilon$\\
$<$\textit{instruction}$>$ $\rightarrow$ \textbf{\red{ID :=}} $<$\textit{\blu{expression}}$>$ \red{\textbf{;}}\\
$<$\textit{instruction}$>$ $\rightarrow$ \textbf{\red{lire (}}$<$\textit{liste\_id}$>$\red{\textbf{);}}\\
$<$\textit{instruction}$>$ $\rightarrow$ \textbf{\red{ecrire (}}$<$\textit{liste\_expressions}$>$\textbf{\red{);}}\\
$<$\textit{instruction}$>$ $\rightarrow$ \textit{if\_block}\\
$<$\textit{liste\_id}$>$ $\rightarrow$ \textbf{\red{ID}} $<$\textit{fin\_liste\_id}$>$\\
$<$\textit{fin\_liste\_id}$>$ $\rightarrow$ \textbf{\red{, ID}} $<$\textit{fin\_liste\_id}$>$\\
$<$\textit{fin\_liste\_id}$>$ $\rightarrow$ $\epsilon$\\
$<$\textit{if\_block}$>$ $\rightarrow$ \textbf{\red{if}} $<$\textit{expression\_booleenne}$>$ \textbf{\red{then \{}}$<$\textit{liste\_instructions}$>$\red{\textbf{\}}} $<$\textit{fin\_if}$>$\\
$<$\textit{fin\_if}$>$ $\rightarrow$ \textbf{\red{else}} \textbf{\red{\{}}$<$\textit{liste\_instructions}$>$\textbf{\red{\}}}\\
$<$\textit{fin\_if}$>$ $\rightarrow$ $\epsilon$\\
$<$\textit{liste\_expressions}$>$ $\rightarrow$ $<$\textit{\blu{expression}}$>$ $<$\textit{fin\_liste\_expression}$>$\\
$<$\textit{liste\_expressions}$>$ $\rightarrow$ $<$\textit{expression\_booleenne}$>$ $<$\textit{fin\_liste\_expression}$>$\\
$<$\textit{fin\_liste\_expression}$>$ $\rightarrow$ \textbf{\red{,}} $<$\textit{\blu{expression}}$>$ $<$\textit{fin\_liste\_expression}$>$\\
$<$\textit{fin\_liste\_expression}$>$ $\rightarrow$ \textbf{\red{,}} $<$\textit{expression\_booleenne}$>$ $<$\textit{fin\_liste\_expression}$>$\\
$<$\textit{fin\_liste\_expression}$>$ $\rightarrow$ $\epsilon$\\
$<$\textit{\blu{expression}}$>$ $\rightarrow$ $<$\textit{\blu{expression}}$>$ \textbf{\red{+}} $<$\textit{\blu{expression}}$>$\\
$<$\textit{\blu{expression}}$>$ $\rightarrow$ $<$\textit{\blu{expression}}$>$ \textbf{\red{-}} $<$\textit{\blu{expression}}$>$\\
$<$\textit{\blu{expression}}$>$ $\rightarrow$ $<$\textit{\blu{expression}}$>$ \textbf{\red{*}} $<$\textit{\blu{expression}}$>$\\
$<$\textit{\blu{expression}}$>$ $\rightarrow$ $<$\textit{\blu{expression}}$>$ \textbf{\red{/}} $<$\textit{\blu{expression}}$>$\\
$<$\textit{\blu{expression}}$>$ $\rightarrow$ $<$\textit{expression\_simple}$>$ $<$\textit{expression\_simple}$>$\\
$<$\textit{\blu{expression}}$>$ $\rightarrow$ $<$\textit{expression\_simple}$>$\\
$<$\textit{expression\_simple}$>$ $\rightarrow$ \textbf{\red{(}}$<$\textit{\blu{expression}}$>$\textbf{\red{)}}\\
$<$\textit{expression\_simple}$>$ $\rightarrow$ \textbf{\red{ID}} \\
$<$\textit{expression\_simple}$>$ $\rightarrow$ \textbf{\red{INT}}\\
$<$\textit{expression\_simple}$>$ $\rightarrow$ \textbf{\red{REAL}}\\
$<$\textit{expression\_simple}$>$ $\rightarrow$ \textbf{\red{STRING}}\\
$<$\textit{expression\_boolenne}$>$ $\rightarrow$ $<$\textit{expression\_boolenne}$>$ \textbf{\red{AND}} $<$\textit{expression\_boolenne}$>$\\
$<$\textit{expression\_boolenne}$>$ $\rightarrow$ $<$\textit{expression\_boolenne}$>$ \textbf{\red{OR}} $<$\textit{expression\_boolenne}$>$\\
$<$\textit{expression\_boolenne}$>$ $\rightarrow$ \textbf{\red{!}}$<$\textit{expression\_boolenne}$>$\\
$<$\textit{expression\_boolenne}$>$ $\rightarrow$ $<$\textit{expression\_simple\_booleenne}$>$\\
$<$\textit{expression\_boolenne}$>$ $\rightarrow$ $<$\textit{\blu{expression}}$>$ \red{\textbf{$>=$}} $<$\textit{\blu{expression}}$>$\\
$<$\textit{expression\_boolenne}$>$ $\rightarrow$ $<$\textit{\blu{expression}}$>$ \red{\textbf{$<=$}} $<$\textit{\blu{expression}}$>$\\
$<$\textit{expression\_boolenne}$>$ $\rightarrow$ $<$\textit{\blu{expression}}$>$ \red{\textbf{$>$}} $<$\textit{\blu{expression}}$>$\\
$<$\textit{expression\_boolenne}$>$ $\rightarrow$ $<$\textit{\blu{expression}}$>$ \red{\textbf{$<$}} $<$\textit{\blu{expression}}$>$\\
$<$\textit{expression\_boolenne}$>$ $\rightarrow$ $<$\textit{\blu{expression}}$>$ \red{\textbf{$==$}} $<$\textit{\blu{expression}}$>$\\
$<$\textit{expression\_simple\_booleenne}$>$ $\rightarrow$ \textbf{\red{TRUE}}\\
$<$\textit{expression\_simple\_booleenne}$>$ $\rightarrow$ \textbf{\red{FALSE}}\\
$<$\textit{expression\_simple\_booleenne}$>$ $\rightarrow$ \textbf{\red{(}}$<$\textit{expression\_booleenne}$>$\textbf{\red{)}}\\	
\end{minipage}}
\end{center}

\section{Regex}
Le fichier .flex contient des expressions régulières permettant de parser correctement le fichier d'entrée. Nous avons eu besoin de définir les regex suivantes :\\
\begin{itemize}
\item un nombre entier : ([1-9][0-9]*) $\vert$ 0
\item un nombre réel : (-$\vert$+)? entier .? [0-9]*([eE]-?[0-9]+)?
\item des expressions booléennes : true$\vert$True$\vert$TRUE$\vert$\#t et false$\vert$False$\vert$FALSE$\vert$\#f
\item un nom de variable en Java : [a-z][a-zA-Z0-9\_]* 
\item une chaîne de caractères : " [\^\space $\backslash$n"]* "
\item des commentaires : (("//"[\^\space $\backslash$n]*) $\vert$ ("/*"[\^\space"*/"]*"*/"))* $\vert$ $\backslash$n* $\vert$ $\backslash$r* $\vert$ ' '* $\vert$ $\backslash$t* 
\end{itemize}

\end{document}