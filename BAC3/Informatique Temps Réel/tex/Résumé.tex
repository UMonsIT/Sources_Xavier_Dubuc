\documentclass{article}

\usepackage{amsmath}
\usepackage{amsfonts}
\usepackage{amssymb}
\usepackage{multicol}
\usepackage{mathenv}

\def\nbOne{{\mathchoice {\rm 1\mskip-4mu l} {\rm 1\mskip-4mu l}
{\rm 1\mskip-4.5mu l} {\rm 1\mskip-5mu l}}}

\usepackage{vmargin}
\setmarginsrb{2.5cm}{2.5cm}{2.5cm}{2.5cm}{0cm}{0cm}{0cm}{0cm}

\usepackage[utf8]{inputenc}

\usepackage[french]{babel}
\selectlanguage{french}

\usepackage{color}
\usepackage{graphicx}
\usepackage{listings}
\definecolor{colKeys}{rgb}{0.75,0,0}
\definecolor{colIdentifier}{rgb}{0,0,0}
\definecolor{colComments}{rgb}{0.75,0.75,0}
\definecolor{colString}{rgb}{0,0,0.7}

\lstset{
basicstyle=\ttfamily\small, %
identifierstyle=\color{colIdentifier}, %
keywordstyle=\color{colKeys}, %
stringstyle=\color{colString}, %
commentstyle=\color{colComments}
}
\lstset{language=java}

% Commandes personnelles %

\definecolor{darkred}{rgb}{0.85,0,0}
\definecolor{darkblue}{rgb}{0,0,0.7}
\definecolor{darkgreen}{rgb}{0,0.6,0}
\definecolor{darko}{rgb}{0.93,0.43,0}
\newcommand{\dred}[1]{\textcolor{darkred}{\textbf{#1}}}
\newcommand{\dgre}[1]{\textcolor{darkgreen}{\textbf{#1}}}
\newcommand{\dblu}[1]{\textcolor{darkblue}{\textbf{#1}}}
\newcommand{\dora}[1]{\textcolor{darko}{\textbf{#1}}}
\newcommand{\gre}[1]{\textcolor{darkgreen}{#1}}
\newcommand{\blu}[1]{\textcolor{darkblue}{#1}}
\newcommand{\ora}[1]{\textcolor{darko}{#1}}
\newcommand{\red}[1]{\textcolor{darkred}{#1}}

\newcommand{\ceil}[1]{\left\lceil #1 \right\rceil}
\newcommand{\cdil}[1]{\left\lfloor #1 \right\rfloor}
\newcommand{\image}[1]{\includegraphics{#1}}
\newcommand{\imageR}[2]{\includegraphics[width=#2px]{#1}}
\newcommand{\imageRT}[2]{\includegraphics[height=#2px]{#1}}
\newcommand{\img}[1]{\begin{center}\includegraphics[width=400px]{#1}\end{center}}
\newcommand{\imag}[1]{\begin{center}\includegraphics{#1}\end{center}}
\newcommand{\imgR}[2]{\begin{center}\includegraphics[width=#2px]{#1}\end{center}}
\newcommand{\imgRT}[2]{\begin{center}\includegraphics[height=#2px]{#1}\end{center}}
\newcommand{\point}[2]{\item \ora{\underline{#1}} : \textit{#2}}
\newcommand{\bfp}[2]{\item \textbf{#1} : \textit{#2}}
\newcommand{\sumin}[3]{\sideset{}{_{i=#1}^{#2}}\sum{#3}}
\newcommand{\sumkn}[3]{\sideset{}{_{k=#1}^{#2}}\sum{#3}}
\newcommand{\stitre}[1]{\noindent\textbf{\underline{#1}} \\}

\DeclareMathAlphabet{\mathpzc}{OT1}{pzc}{m}{it}

\title{\textbf{\textcolor{darkblue}{Informatique Temps Réel - Résumé Janvier 2010.}}}
\author{\textit{Dubuc Xavier}}

\begin{document}

\maketitle

\hbox{\raisebox{0.4em}{\vrule depth 0.4pt height 0.4pt width 10cm}}

\tableofcontents

$ $ \\
\hbox{\raisebox{0.4em}{\vrule depth 0.4pt height 0.4pt width 10cm}}

\newpage

\section*{Introduction}

Il s'agit d'un cours résumé servant uniquement \textbf{à l'étude}, car à l'examen, même si c'est «avec le 
cours», vous ne pourrez prendre ce pdf. Il est aussi à rappeller la répartition des points : \\
\begin{itemize}
\item \dred{$20\%$} pour l'exercice de la taverne,
\item \dred{$40\%$} pour un exercice sur machine à l'examen (en rapport avec ce qui a été vu en 
\textbf{TP}),
\item \dred{$40\%$} pour la partie de l'examen concernant l'application de la théorie (cours ouvert).
\end{itemize}

\section{Chapitre 1}

\subsection{Definitions}

Un \textbf{système temps réel} est un système dans lequel l'exactitude des applications ne dépend pas 
seulement de l'exactitude des résultats mais aussi du temps auquel ce résultat est produit. Si les 
contraintes temporelles de l'application ne sont pas respectées, on parle de défaillance du système. Il est 
donc essentiel de garantir le respect des containtes temporelles du système. \\

Le système recueille des informations via des \textbf{capteurs} et il agit sur celui-ci via des 
\textbf{actuateurs}. Un exemple simple est un controleur de débit, on place un \textbf{capteur} calculant
le débit à l'entrée d'un tuyau et plus loin on place un \textbf{actuateur} (une valve) servant à controler
l'eau qui sort du tuyau. \\

Le système est un ensemble de \textbf{tâches} autonomes s'exécutant concurremment, le code d'une
\textbf{tâche} correspondant à un bloc modulaire d'instructions. Chaque \textbf{tâche} possède son propre 
flot de contrôle à l'exécution et peut communiquer et partager ses ressources avec d'autres \textbf{tâches} 
du système. On définira donc des tâches associées à un ou des évènements, une ou des réactions, une entité
externe à calculer, un traitement particulier, ...

L'\textbf{ordonnanceur} est le nom donné au module qui se charge de gérer l'enchaînement et la concurrence
des tâches en optimisant l'occupation de l'unité centrale.

\subsection{Fonctionnement}

\begin{enumerate}
\item \underline{Fonctionnement cyclique}
\begin{lstlisting}
A chaque top horloge faire
   - Lecture de la memoire des entrees
   - Calcul des ordres a envoyer au procede
   - Emission des ordres
\end{lstlisting}

\item \underline{Fonctionnement évènementiel}

\begin{lstlisting}
while(1) {
   Attendre les evenements signales par interruption
   A chaque interruption faire {
      - Lecture information arrivee
      - Activation du traitement correspondant
      - Emission des ordres issus du traitement
   }
}
\end{lstlisting}

Que faire si une interruption survient alors que le système est en train de traiter une interruption 
précédente ? \\
$\rightarrow$ notion de priorité des interruptions \\
$\rightarrow$ notion de tâche associée à une ou plusieurs interruptions \\
$\rightarrow$ mécanisme de préemption (réquisition du processeur) et de reprise de tâche au retour du «vol»
du \textbf{CPU}. \\
$\rightarrow$ gestion de l'exécution concurrente des tâches (ordonnancement) \\

\dred{Un système temps-réel fonctionnera de cette façon.}

\item \underline{Fonctionnement mixte} \\
Mélange des 2 précédents.
\end{enumerate}

\subsection{Caractéristiques}

Les caractéristiques nécessaires à un système T.R. sont les suivantes : 
\begin{itemize}
\item \textbf{Fiabilité} (systèmes critiques)
\item \textbf{Prédictibilité} (garantir le respect des contraintes temporelles)
\item \textbf{Préemptibilité des tâches} (nécessaire pour la prédictibilité)
\item \textbf{Gestion poussée des communications inter-tâches}
\item \textbf{Prise en compte du non-déterminisme d’ordre d’exécution des tâches} ; on considère souvent le 
«worst-case» (cas au pire) pour déterminer un temps de réaction.
\end{itemize}

\section{Chapitre 2 - Développement d'un système temps réel}

\subsection{Capturer les exigences}

Analyse des besoins :
\begin{itemize}
\item Les \textbf{entrées} : \textit{Acquisition de grandeurs par des senseurs.}
\item Les \textbf{sorties} : \textit{Commande de l’environnement par les actuateurs.}
\item Les \textbf{traitements} : \textit{Mise à jour des sorties en fonction des entrées pour 
contrôler/commander un processus physique.}
\item La \textbf{visualisation} : \textit{Présenter à l’utilisateur l’état du procédé et de sa gestion.}
\item Les \textbf{communications} : \textit{Partager des données avec les systèmes de contrôle commande 
voisins.}
\item Les \textbf{sauvegardes} \textit{des états pour des reprises d’exploitation ou des analyses du système 
a posteriori.}
\end{itemize}

\begin{enumerate}
\item \ora{\underline{Les tâches d'entrées}} \\
Acquérir une donnée physique n'est pas trivial, il y a beaucoup de données de types différents et de 
capteurs de différentes natures et/ou ayant des fonctionnements différents. 
\item \ora{\underline{Les tâches de sorties}} \\
Il y a différents types d'actuateurs permettant au système d'agir sur l'environnement, 
\begin{itemize}
\item \textbf{Binaires} : soit on active soit on active pas, soit on ouvre soit on ouvre pas (enclenchement
d'un ventilateur, ouverture d'une valve, ...),
\item \textbf{Discrètes} : sur quel type de bus,
\item \textbf{Continues} : des consignes précises quant à une modification (consigne de vitesse, 
température,...).
\end{itemize}
\item \ora{\underline{Les tâches de traitements}} \\
Les tâches qui effectuent des analyses, des calculs sur les entrées afin d'en dégager les sorties.
\item \ora{\underline{Les tâches d'interface utilisateur}} \\
Gérer un pupitre de commande permettant de visualiser l'état du procédé, les valeurs des consignes et les
alarmes ainsi que d'agir sur les consignes et les commandes.
\item \ora{\underline{Les tâches de communication}} \\
Tâches gérant la communication entre les tâches, elles sont contraintes par des contraintes logicielles et 
matérielles. (différents moyens de communiquer, différents moyens de formater les informations,...)
\item \ora{\underline{Les tâches de sauvegarde}} \\
Elles sauvegardent les états du systèmes régulièrement afin de permettre d'analyser son fonctionnement, les
erreurs, de permettre des reprises et mises en route ainsi que d'améliorer les performances.
\end{enumerate}

\noindent Il faut maintenant parvenir à mettre en oeuvre l'ensemble de ces tâches en garantissant la 
\textbf{synchronisation}, les \textbf{transferts de données} et le \textbf{partage des ressources}. Pour ce
faire, il y a 2 modèles :
\begin{itemize}
\item \red{Synchrone}
	\begin{itemize}
	\item Les évènements émis par le procédé peuvent être \textit{différés}.
	\item Les évènements sont perçus comme \textit{immédiats} par l'informatique.
	\item Les évènements externes sont \textbf{synchronisés} avec les tâches.
	\item Non préemptif, ordonnancement \textit{à priori}, \textbf{séquenceur}.
	\end{itemize}
\item \red{Asynchrone}
	\begin{itemize}
	\item Les évènements émis par le procédé sont traités immédiatement $\rightarrow$ interruptions.
	\item Les évènements sont perçus comme \textit{immédiats} par l'informatique.
	\item Les évènements externes ne sont \textbf{pas synchronisés} avec les tâches.
	\item Préemptif, ordonnancement \textit{en ligne}, noyau \textbf{temps réel}.
	\end{itemize}
\end{itemize}

\subsection{Méthodologie stricte de développement}

Il convient d'adopter une méthodologie pour le développement d’applications informatiques :
\begin{itemize}
\item L’architecture, la maintenabilité, la lisibilité, le suivi des spécifications, l’évolubilité sont 
augmentés.
\item Nécessité de penser au préalable aux tests unitaires (par modules) et globaux.
\end{itemize}

Une méthodologie répandue, le \textbf{cycle en V} mais pas forcément adaptée au contrôle-commande :

\begin{lstlisting}
Specification                               Validation
      Conception Preliminaire         Integration
          Conception Detaillee   Tests Unitaires
                            Codage
\end{lstlisting}

Moins problématique, le \textbf{cycle en W} :

\begin{lstlisting}
Specification              Simulation                            Validation
     Conception     Integration  Conception Adaptee         Integration (avec noyau RT)
                  Tests                                  Tests
             Codage                       Codage (croise)
\end{lstlisting}

Nous allons utiliser 2 autres approches, plus complètes. Nous les développons ci-dessous.

\subsubsection{Structured Analysed Real Time (\textit{\red{SART}})}

\stitre{\'{E}léments graphiques}

\imgR{ITR_001.png}{400}

\stitre{Les données}

\imgR{ITR_002.png}{400}

\stitre{Les processus fonctionnels}

\imgR{ITR_003.png}{150}

\stitre{Stockage}

\imgR{ITR_004.png}{400}

\stitre{Terminaisons}

\imgR{ITR_005.png}{300}

\stitre{Les processus de contrôle}

Le processus de contrôle représente la logique de pilotage des processus fonctionnels. Les processus fonctionnels 
fournissent tous les évènements aux processus de contrôle qui gèrent les évènements d'(de) (dés)activation des
processus fonctionnels.

\imgR{ITR_006.png}{75}

\stitre{Les flots de contrôle}

Les flots de contrôle transportent les évènements qui conditionnent, directement ou indirectement, l’exécution 
des processus fonctionnels.

\imgR{ITR_007.png}{400}

\noindent Il y a 3 évènements prédéfinis : \dred{E} \textit{enable} (\textbf{activation}), \dred{D} 
\textit{Disable} (\textbf{désactivation}) et \dred{T} \textit{Trigger} (\textbf{enclenchement}).

\subsubsection{Design Approach for Real Time System (\textit{\red{DARTS}})}

Cette approche permet de mettre l'accent sur la synchronisation et la communication entre les tâches.\\

\stitre{Relations entre tâches}

Une tâche ne peut débuter que lorsqu'une autre tâche s'est exécutée en totalité ou en partie :

\begin{multicols}{2}
\textbf{Asynchrone ou unilatérale} : \textit{une seule tâche est bloquée/en attente,}
\imgR{ITR_008.png}{150}
\end{multicols}

\begin{multicols}{2}
\textbf{Synchrone ou bilatérale} : \textit{les 2 tâches doivent atteindre un point de rendez-vous,}
\imgR{ITR_009.png}{150}
\end{multicols}

\stitre{Activation des tâches}

\begin{itemize}
\item Tâche matérielle :
	\begin{itemize}
	\item \begin{multicols}{2}
			périodique : horloge temps réel : \textbf{HTR}\textit{(durée)}
			\imgR{ITR_010.png}{150}
		  \end{multicols}
  	\item \begin{multicols}{2}
			Non-périodique : interruption : \textbf{IT}\textit{(nom)} ou chien de garde : 
			\textbf{CG}\textit{(nom)}
			\imgR{ITR_011.png}{150}
		  \end{multicols}
	\end{itemize}
\item Tâche logicielle : boîte aux lettres \\
\imgR{ITR_012.png}{225}
\end{itemize}

\stitre{Tâches}

\begin{multicols}{2}
Même définition que dans \dred{SART} pour les processus fonctionnel mis à part qu'un signal d'activation est 
nécessaire en entrée.
\imgR{ITR_013.png}{150}
\end{multicols}

\stitre{Module de traitement}

\begin{multicols}{2}
Il s'agit de programmes spécifiques appelés par une ou plusieurs tâches, ils sont ré-entrants et permettent 
d'alléger le code des tâches tout en améliorant la structure.
\imgR{ITR_014.png}{150}
\end{multicols}

\stitre{Synchronisation \& Communication}

\begin{itemize}
\item \dred{Synchronisation}
	\begin{itemize}
	\item \begin{multicols}{2}
		Asynchrone simple, multiple de type \textit{«OU»}
		\imgR{ITR_015.png}{110}
		\end{multicols}
	\item \begin{multicols}{2}
		Synchrone
		\imgR{ITR_016.png}{50}
		\end{multicols}
	\end{itemize}
\item \dred{Communication} : boîte aux lettres (\textbf{BAL})
	\begin{itemize}
	\item \textbf{BAL} bloquante en écriture
		\begin{itemize}
		\item \begin{multicols}{2}
			1 message
			\imgR{ITR_017.png}{50}
			\end{multicols}
		\item \begin{multicols}{2}
			FIFO à n messages, simple ou multiples
			\imgR{ITR_018.png}{100}
			\end{multicols}
		\item \begin{multicols}{2}
			FIFO à n messages et priorités
			\imgR{ITR_019.png}{50}
			\end{multicols}
		\end{itemize}
	\item \textbf{BAL} non bloquante ou à écrasement
		\begin{itemize}
		\item \begin{multicols}{2}
			1 message
			\imgR{ITR_020.png}{50}
			\end{multicols}
		\item \begin{multicols}{2}
			FIFO à n messages, simple ou multiples
			\imgR{ITR_021.png}{100}
			\end{multicols}
		\end{itemize}	
	\end{itemize}
\end{itemize}

\stitre{Stockage}

\begin{multicols}{2}
Module de données avec protection des accès par exclusion mutuelle
\imgR{ITR_022.png}{200}
\end{multicols}

\section{Chapitre 3 - Processus \& Threads}

\textit{Useless de résumer une notion vue \& revue 42 fois en 3 ans ...}

\section{Chapitre 4 - L'ordonnancement}

\subsection{Table de correspondance des notations}

\begin{center}
	\begin{tabular}{|*{2}{c|}}
	\hline
	\dred{Notation} & \dred{Signification} \\
	\hline
	$\tau_i$ & tâche de nom $\tau_i$\\
	\hline
	$\tau_{i,j}$ & job $j$ de la tâche de nom $\tau_i$\\
	\hline
	$T_i$ & période de la tâche$\Phi$\\
	\hline
	$r_i$ & temps de \textit{release} ou d'arrivée dans le pool des tâches\\
	\hline
	$s_i$ & temps de démarrage réel\\
	\hline
	$\Phi_i$ & temps de démarrage de la tâche (dans le cas de séparation en plusieurs jobs)\\
	\hline
	$C_i$ & temps d'exécution au pire\\
	\hline
	$d_i$ & deadline absolu\\
	\hline
	$D_i$ & deadline relatif (= $d_i-r_i$)\\
	\hline
	$f_i$ & temps maximal de fin d'une tâche (\textbf{contrainte} $f_i\leq d_i$)\\
	\hline
	$\max{(0,f_i-d_i)}$ & avance ou retard de la tâche \\
	\hline
	$c_i(t)$ & temps maximal restant à exécuter \\
	\hline
	$d_i-t-c_i(t)$ & laxité, c'est-à-dire une période creuse d'exécution (\textbf{slack}) \\
	\hline
	$p_i$ & Priorité de la tâche \\
	\hline
	$R_i$ & Temps de réponse \\
	\hline
	$U_i$ & Taux d'utilisation maximale du processeur par période par la tâche ($U_i = \frac{C_i}{T_i}$) \\
	\hline
	$U_p$ & Taux d'utilisation totale du processeur $\left(U_p = \sumin{1}{n}{U_i}\right)$ \\
	\hline
	\end{tabular}
\end{center}

\imgR{ITR_023.png}{200}

Dans le cours sont développées principalement 2 manières d'ordonnancer,

\subsection{Tâches périodiques}

\subsubsection{EDF : Earliest Deadline First}

\begin{center}$\boxed{p_i = \dfrac{1}{d_i}}$ \textit{(dynamique)}\end{center}

Des tâches sont ordonnançables si : $\boxed{\sideset{}{_{i=1}^n}\sum{\dfrac{C_i}{T_i}} \leq 1}$

Demande du processeur dans $[0,L]$ : $\boxed{g(0,L)\sumin{1}{n}{\cdil{\dfrac{L-D_i+T_i}{T_i}}C_i}}$ \\

\noindent\dred{L'ordonnancement est dès lors acceptable si $\forall L \in D,\ g(0,L) \leq L$}.\\
\textit{où $D =\{d_k|d_k\leq \min{(H,L^*)}\}$, $H = lcm(T_1,...,T_n)=ppcm(T_1,...,T_n)$ et
$L^*=\dfrac{\sumin{i}{n}{(T_i-D_i)U_i}}{1-U_p}$}

\subsubsection{RM (Rate Monotonic) (et DM (Deadline Monotonic))}

\begin{center}$\boxed{p_i = \dfrac{1}{D_i}}$ \textit{(statique, souvent $D_i=T_i$)}\end{center}

Des tâches sont ordonnançables si : $\boxed{\sideset{}{_{i=1}^n}\sum{\dfrac{C_i}{T_i}} \leq n(2^{\frac{1}
{n}}-1)}$

Temps de réponse : $\boxed{R_i = C_i + \sumkn{1}{i-1}{\ceil{\dfrac{R_i}{T_k}} C_k}}$ \\
Afin de calculer ce temps de réponse $R_i$, il existe une manière itérative d'y parvenir : \\
$R_{i,0} = C_i \\
 R_{i,s} = C_i + \sumin{k=1}{i-1}{\ceil{\dfrac{R_{i,s-1}}{T_k}} C_k}$ \\
 
Calcul que l'on itère jusqu'à ce que $R_{i,s} = R_{i,s-1}$ ce qui signifie que l'on a trouvé le temps de réponse
ou que $R_{i,s} > D_i$ qui signifie que le temps de réponse est plus long que le deadline relatif. \\
 
\noindent\dred{L'ordonnancement est dès lors acceptable si $R_i \leq D_i\ \forall$ tâche $\tau_i$.}
 
\end{document}